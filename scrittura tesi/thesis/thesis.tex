%
% Tesi D.S.I. - modello preso da
% Stanford University PhD thesis style -- modifications to the report style
%
%%%%%%%%%%%%%%%%%%%%%%%%%%%%%%%%%%%%%%%%%%%%%%%%%%%%%%%%%%%%%%%%%%%%%%%%%%%
%                                                                         %
%			TESI DOTTORATO                                                   %
%			______________                                                   %
%                                                                         %
%			AUTORE: Elena Pagani                                             %
%                                                                         %
%			Ultima revisione: 7.X.1998                                       %
%           correzioni atrent                                             %
%%%%%%%%%%%%%%%%%%%%%%%%%%%%%%%%%%%%%%%%%%%%%%%%%%%%%%%%%%%%%%%%%%%%%%%%%%%
%
%
\documentclass[a4paper,12pt]{report}
%    \renewcommand{\baselinestretch}{1.6}      % interline spacing
%
% \includeonly{}
%
%			PREAMBOLO
%
\usepackage[a4paper]{geometry}
\usepackage{amssymb,amsmath,amsthm}
\usepackage{graphicx}
\usepackage{url}
\usepackage[pdfa]{hyperref}
\usepackage{epsfig}
\usepackage[english]{babel}
\usepackage{setspace}
\usepackage{tesi}
\usepackage{algorithm}
\usepackage{algcompatible}
%\usepackage{algorithmicx}
\usepackage{comment}
\usepackage[section]{placeins}
\usepackage[a-1b]{pdfx}
\usepackage[utf8]{inputenc}
\usepackage{csquotes}
\usepackage{tikz}
\usepackage{subcaption}
\usepackage{tabularx}
\usepackage{longtable}
\usepackage{booktabs}
\usepackage{mathtools}
%\usepackage{float}
%\restylefloat{table}
\newif\ifshowtableLinear
\showtableLinearfalse % Cambia in \showtabletrue per mostrare la tabella

\setcounter{secnumdepth}{5}

%
\newtheorem{myteor}{Teorema}[section]
%
\newenvironment{teor}{\begin{myteor}\sl}{\end{myteor}}
%
\usepackage[
backend=biber,
style=alphabetic,
sorting=ynt
]{biblatex}
\addbibresource{biblio.bib}
\usepackage{subfiles}
%
%			TITOLO
%
\begin{document}

\includegraphics[width=\textwidth]{thesis/picture/tesiSCIENZE_TECNOLOGIE.jpg}
\title{Methods for traffic management in urban air mobility}
\author{Filippo Magi}
\dept{Corso di Laurea in Informatica Magistrale} 
\anno{2023-2024}
\matricola{17305A}
\relatore{Prof. Alberto Ceselli}
\correlatore{Prof. Claudia D'Ambrosio}
%
%        \submitdate{month year in which submitted to GPO}
%		- date LaTeX'd if omitted
%	\copyrightyear{year degree conferred (next year if submitted in Dec.)}
%		- year LaTeX'd (or next year, in December) if omitted
%	\copyrighttrue or \copyrightfalse
%		- produce or don't produce a copyright page (false by default)
%	\figurespagetrue or \figurespagefalse
%		- produce or don't produce a List of Figures page
%		  (false by default)
%	\tablespagetrue or \tablespagefalse
%		- produce or don't produce a List of Tables page
%		  (false by default)
% 
%			DEDICA
%
\beforepreface
\prefacesection{}
        {\hfill \Large {\sl dedicato a \dots}}
% 
%			PREFAZIONE
%
\prefacesection{Preface}
\subfile{thesis/sections/intro/preface}
%
%
%			ORGANIZZAZIONE
\section*{Thesis Organization}
\label{organizzazione}

This thesis is organized as follows:
\begin{itemize}
    \item \textbf{Chapter 1}: We introduce the problem and review previous works.
    \item \textbf{Chapter 2}: We present a general-use linear model to define the problem, highlight the differences with Pelegrin's work, and describe how we generate instances for testing.
    \item \textbf{Chapter 3}: We introduce a modified model for solving Pelegrin's tactical deconfliction with some fixed paths.
    \item \textbf{Chapter 4}: We present a heuristic approach for solving Pelegrin's tactical deconfliction while allowing path modifications.
    \item \textbf{Chapter 5}: We analyze the results obtained from both the mathematical model and the heuristic for solving tactical deconfliction.
\end{itemize}
% 
% 

\afterpreface
%			CAPITOLO 1: dshjkfg
\chapter{Introduction}
\subfile{thesis/sections/intro/Introduction}
\subfile{thesis/sections/intro/pelegrinSummary}
\subfile{thesis/sections/intro/PortoleauSummary}

%
%
\chapter{Definition of the problem}
%\newpage
\section{The model}\label{sec:modelDescription}
We define the problem through its mathematical model.
First, we introduce the decision variables, followed by an explanation of the input data.
Finally, we present the constraints and describe the techniques used to linearize the nonlinear ones.
\subfile{thesis/sections/generalModel/variablesAndData}
\subfile{thesis/sections/generalModel/constraints}
\subfile{thesis/sections/generalModel/linearization}
\newpage
\subfile{thesis/sections/mercedesChanges}
\newpage
\section{Testing maps}\label{sec:maps}
\subfile{thesis/sections/maps/topologies}
\subfile{thesis/sections/maps/generationOfIstances}

\chapter{Exact Method}\label{chap:ExactAlgo}
\subfile{thesis/sections/PelegrinModification}
\chapter{Heuristic Algorithms}\label{chap:HeuristicAlgo}
\subfile{thesis/sections/heuristic}
%Da controollare risultati computazionali
\chapter{Computational results}\label{chap:results}
In this chapter, we present the results obtained for both the model and the heuristic.  
For a complete overview of all instance results, refer to \cite{Magi_Methods_for_Traffic_2025}.  
In both cases, we performed our calculations on the Atlas server at Polytechnique, equipped with an Intel\textregistered~Xeon\textregistered~Platinum 8362 CPU @ 2.80GHz and 2TiB of system memory.
\subfile{thesis/sections/conclusions/PelegrinConclusion}
\subfile{thesis/sections/conclusions/HeuristicConclusion}
\chapter{Conclusion}\label{chap:conclusion}
\subfile{thesis/sections/conclusions/Conclusion}
%
%			BIBLIOGRAFIA

\printbibliography

\appendix 
\chapter{Models} 
\section{The non linearized model}\label{sec:wholeModel}
\subfile{thesis/models/wholeModel} 
\section{The linearized model}\label{sec:linearModel}
\subfile{thesis/models/linearModel}
\section{Pelegrin Model with Path Changing Capability}\label{sec:MercedesModifedModel}
\subfile{thesis/models/MercedesModel}
\chapter{The algorithms}
\section{The math-heuristic}\label{alg:matheuristic}
\subfile{thesis/pseudocode/matheuristic}
\chapter{Tables}
\section{Mercedes}
%number of variables
\subsection{Number of variables for fixed Flights}
\subfile{thesis/tables/mercedes/variables/SummaryTablesFixed}
\subsection{Number of variables for free Flights}
\subfile{thesis/tables/mercedes/variables/SummaryTablesFree}
\subsection{Comparison between number of variables for free and fixed flights}
\subfile{thesis/tables/mercedes/variables/ComparisonTables}
%times and results
\subsection{Results and calculation time for fixed Flights}
\subfile{thesis/tables/mercedes/times/SummaryTablesFixed}
\subsection{Results and calculation time for free Flights}
\subfile{thesis/tables/mercedes/times/SummaryTablesFree}
\subsection{Comparison between results and calculation time for free and fixed flights}
\subfile{thesis/tables/mercedes/times/ComparisonTablesResults}
\section{heuristic}
\subsection{Summary tables}
\subfile{thesis/tables/heuristic/times/SummaryTables}\label{chap:heuristic:tables}
\subsection{Comparison between heuristic and no fixed path}
\subfile{thesis/tables/heuristic/times/comparison}
\end{document}


 
