\documentclass[../../thesis.tex]{subfiles}

\begin{document}
Urban Air Mobility (UAM) refers to air transport systems designed to move people and goods within and around densely populated urban areas. Ensuring safety and maximizing efficiency are crucial challenges in this field.  

In \cite{pelegrin-2023}, a mathematical formulation for reoptimization in tactical deconfliction is presented. Tactical deconfliction is one of the layers into which air traffic management is typically divided. The first layer addresses strategic deconfliction, which is computed prior to departure. The second layer focuses on tactical deconfliction, aiming to maintain separation between ongoing flights. The third and final layer is dedicated to collision avoidance, leveraging onboard technologies to prevent imminent collisions.

This work primarily concentrates on the second layer, which seeks to prevent loss of separation due to random variations, priority traffic, or non-cooperative intruders. Additionally, three urban topologies are proposed for testing—Grid, Airport, and Metroplex—which will be described in detail in Section \ref{sec:maps}.  

In \cite{portoleau-2024}, a two-stage robust algorithm for the UAM flight scheduling problem is introduced. The algorithm is based on the Adversarial Benders Decomposition Method, which enables the use of heuristics for solving subproblems. One of the proposed heuristics involves fixing the flight order and solving the approximated subproblem either as a quadratic program or as a linear program with an increasing piecewise-linear objective function.  

We propose a new mathematical model to address the deconfliction problem, incorporating the capability to modify the path for a subset of flights. Moreover, if this subset is too large—such as the entire set of flights—we introduce a math-heuristic to efficiently tackle the problem.  

The introduction of the capability to change the path aims to improve results and obtain feasible solutions in cases where a fixed-path approach would otherwise be infeasible.
%Urban air mobility (UAM) refers to air transport systems designed to move people and goods by air within and around dense city areas. It's crucial to develop methods that ensure safety and maximize efficiency.
%In \cite{pelegrin-2023} a mathematical formulation for reoptimization is presented.
%The proposed algorithm is divided in three layers. In the first one the strategic deconflict is calculated prior to departure.
%The second layer focuses on tactical deconfliction, with the goal of maintain separation between ongoing traffic.
%As third and last layer is dedicated to collision avoidance, utilizing onboard technologies to evade imminent collision.
%The paper concentrate on the second layer, which aims to avoid loss of separation due to random variations, priority traffic or non-cooperative intruder.
%Additionally three urban topologies are proposed  for testing, that are the Grid, Airport and Metroplex, that will be described  in detail in section \ref{sec:maps}.\\
%In \cite{portoleau-2024} a two-stage robust algorithm for the scheduling problem of UAM flight is presented.
%The algorithm is based on Adversarial Bender Decomposition Method, which allows to solve the subproblems with heuristics.
%One of the proposed heuristic involves fixing the order and solving the approximated subproblem with either a quadratic problem or with a linear problem with increasing piecewise-linear objective function.
%We propose a new mathematical model to solve the deconfliction problem, incorporating the capability to modify the path for a subset of flights.
%Moreover, if this subset is too large—such as the entire set of flights—we introduce a math-heuristic to efficiently address the problem.
\end{document}